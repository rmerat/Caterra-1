\chapter{Conclusion}

In conclusion, the crop row detection algorithm developed in this report utilizes simple image processing techniques such as color clustering, vanishing point calculation, the Hough transform, and RANSAC. \\

The algorithm is robust to different illuminations, growth stages, and points of view. Since simple computer vision techniques were used, it is also generalizable and is not limited to a single type of plant. \\

The algorithm was found to be accurate for straight crop rows, especially for images with low to medium vegetation density. It achieved a median accuracy of 82\% when tested on a dataset of various images, demonstrating its effectiveness in detecting crop rows in images.  \\

In the future, this algorithm could be further improved by using additional techniques such as machine learning and deep learning models, or LiDAR for very bushy regions. Overall, the algorithm shows promise in being a useful tool for precision agriculture and crop monitoring, but further optimization is needed to make it real-time.